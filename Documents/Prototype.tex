% Created 2019-12-08 Sun 15:41
% Intended LaTeX compiler: pdflatex
\documentclass[11pt]{article}
\usepackage[utf8]{inputenc}
\usepackage[T1]{fontenc}
\usepackage{graphicx}
\usepackage{grffile}
\usepackage{longtable}
\usepackage{wrapfig}
\usepackage{rotating}
\usepackage[normalem]{ulem}
\usepackage{amsmath}
\usepackage{textcomp}
\usepackage{amssymb}
\usepackage{capt-of}
\usepackage{hyperref}
\author{Alice Ip, Kexin Liu, Lily Lau, Meijing Li}
\date{December 08, 2019}
\title{CS/SE 3GC3 - Computer Graphics Final Project}
\hypersetup{
 pdfauthor={Alice Ip, Kexin Liu, Lily Lau, Meijing Li},
 pdftitle={CS/SE 3GC3 - Computer Graphics Final Project},
 pdfkeywords={},
 pdfsubject={},
 pdfcreator={Emacs 26.3 (Org mode 9.1.9)}, 
 pdflang={English}}
\begin{document}

\maketitle

\section*{Project Description}
\label{sec:org7b3eb19}
A cooking game that takes place in a 3-D simulating kitchen,
 where the player selects a recipe and must interact with
 ingredient objects and tool objects and follow the instructions
 given to create the dish.

\section*{Keyboard Commands}
\label{sec:org512dd7d}
\begin{itemize}
\item q - quit
\item 1 - View ingredients/objects to make salad recipe (default view)
\item 2 - View ingredients/objects to make curry recipe
\item 3 - View ingredients/objects to make steak recipe
\item up arrow - zoom out
\item down arrow - zoom in
\item left arrow - rotate left around kitchen counter
\item right arrow - rotate right around kitchen counter
\end{itemize}

\section*{Features from Prototype}
\label{sec:orgb2992fb}

One of the features that we have implemented is the room environment,
 that consists of a floor, walls, and lighting . Another implementation
 in our prototype is a obj reader function
 in the ingredients.cpp file that reads the obj file path passed to it,
 and extracts the appropriate information (mesh, texture, normals,
 faces) into c++ vectors. These vectors are then used to load the
 objects into the room. The appropriate obj files for the
 ingredients/tools/textures were obtained from several sources
 (that are cited in the readme file), and edited as appropriate 
for our needs. 

Summed Up:
\begin{itemize}
\item Basic room with floors, walls, lighting
\item Basic meshes for each food object and tool object, and kitchen counter
\item .obj file parser
\item Loading object meshes and textures into the room
\end{itemize}

\section*{New Features}
\label{sec:org7c8059f}

\subsection*{Key Features (Advanced Graphics Features)}
\label{sec:org89eb218}
\begin{itemize}
\item Lighting [5\%]
\item Textures [10\%]
\item Ray Casting [10\%]
\item Non-geometric primitives(bitmaps, pixel maps) [10\%] 
\begin{itemize}
\item sdfsdfsd
\end{itemize}
\item sdfsdfsd
\end{itemize}

\subsection*{Other Features}
\label{sec:org128895f}
Some of the features that were implemented in the final implementation include the
 application of textures and materials on the objects. Depending on which recipe
is chosen by the user, the appropriate instructions for the recipe and only the 
necessary ingredients for that object are loaded in. Once a recipe is chosen, a 
timer is displayed in the window indicating remaining time for that recipe. The
scoring system is based on the amount of time left when the user completes the 
recipe.

 In order to perform
 actions on the ingredients, we will implement ray casting. A 3D ray will be
 projected from the mouse position on click. This will allow the user to
 select an ingredient or tool, apply a tool to an ingredient (knife to
 orange for fruit salad recipe) or apply an ingredient to a tool (cut potatoes
 to pot for curry recipe). Functions will be constructed to determine 
ingredient state changes if the appropriate tool is used on it (e.g. If a
 knife is used on the orange, the orange will automatically change to the "cut
" orange mesh.) As well, when the correct combinations of ingredients are put
 together, the final recipe object will appear and the game will end. At the
 bottom left corner of the screen, the instructions of the recipe will be displayed.

\subsection*{Recipe Details}
\label{sec:org2182047}

\begin{itemize}
\item Curry
\begin{itemize}
\item Objects: Potato, Tomato, Onion, Knife, Pot
\item Use knife on Potato (whole)
\item Use Potato (cut) on Pot
\item Use knife on Tomato (whole)
\item Use Tomato (cut) on Pot
\item Use knife on Onion (whole)
\item Use Onion (cut) on Pot
\item Curry Complete
\end{itemize}

\item Fruit Salad
\begin{itemize}
\item Objects: Mango, Orange, Banana, Knife
\item Use Knife on Mango (whole)
\item Use Mango (cut) on Bowl
\item Use Knife on Orange (whole)
\item Use Orange (cut) on Bowl
\item Use Knife on Banana (whole)
\item Use Banana (cut) on Bowl
\item Fruit Salad Complete
\end{itemize}

\item Steak
\begin{itemize}
\item Objects: Steak, Pan
\item Use Steak on Pan
\item Wait 10 Seconds
\item Steak Complete
\end{itemize}
\end{itemize}
\end{document}
