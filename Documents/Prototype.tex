% Created 2019-11-13 Wed 22:36
% Intended LaTeX compiler: pdflatex
\documentclass[11pt]{article}
\usepackage[utf8]{inputenc}
\usepackage[T1]{fontenc}
\usepackage{graphicx}
\usepackage{grffile}
\usepackage{longtable}
\usepackage{wrapfig}
\usepackage{rotating}
\usepackage[normalem]{ulem}
\usepackage{amsmath}
\usepackage{textcomp}
\usepackage{amssymb}
\usepackage{capt-of}
\usepackage{hyperref}
\author{Alice Ip, Kexin Liu, Lily Lau, Meijing Li}
\date{November 10, 2019}
\title{CS/SE 3GC3 - Computer Graphics Final Project}
\hypersetup{
 pdfauthor={Alice Ip, Kexin Liu, Lily Lau, Meijing Li},
 pdftitle={CS/SE 3GC3 - Computer Graphics Final Project},
 pdfkeywords={},
 pdfsubject={},
 pdfcreator={Emacs 26.3 (Org mode 9.1.9)}, 
 pdflang={English}}
\begin{document}

\maketitle

\section*{Project Description}
\label{sec:org20b9486}
A cooking game that takes place in a 3-D simulating kitchen, where the player selects a recipe and must interact with ingredient objects and tool objects and follow the instructions given to create the dish.

\section*{Current Features and Commands}
\label{sec:org2a7d9f6}
One of the features that we have implemented is the room environment, that consists of a floor, walls, lighting and materials. Another implementation in our prototype is a obj reader function in the ingredients.cpp file that reads the obj file path passed to it, and extracts the appropriate information (mesh, texture, normals, faces) into c++ vectors. These vectors are then used to load the objects into the room. The appropriate obj files for the ingredients/tools/textures were obtained from several sources (that are cited in the readme file), and edited as appropriate for our needs. 

\subsection*{Keyboard Commands}
\label{sec:org1222e5e}
\begin{itemize}
\item q - quit
\item 1 - View ingredients/objects to make salad recipe (default view)
\item 2 - View ingredients/objects to make curry recipe
\item 3 - View ingredients/objects to make steak recipe
\item up arrow - zoom out
\item down arrow - zoom in
\item left arrow - rotate left around kitchen counter
\item right arrow - rotate right around kitchen counter
\end{itemize}

\section*{Features To Be Implemented/Fixed}
\label{sec:orgcbba231}
Some of the features that will be implemented in the future include the application of textures and materials on the objects. As well, we hope to optimization the object loading, to load only the necessary ingredients/tools required by the recipe the user selects. For example, if the user selects the curry option, we will only load the curry ingredients. In addition, there will be a timer displayed in the window that will inform the user of the remaining time to construct the recipe. A scoring system will be created, and also displayed when the timer runs out. In order to perform actions on the ingredients, we will implement ray casting. A 3D ray will be projected from the mouse position on click. This will allow the user to select an ingredient or tool, apply a tool to an ingredient (knife to orange for fruit salad recipe) or apply an ingredient to a tool (cut potatoes to pot for curry recipe). Functions will be constructed to determine ingredient state changes if the appropriate tool is used on it (e.g. If a knife is used on the orange, the orange will automatically change to the "cut" orange mesh.) As well, when the correct combinations of ingredients are put together, the final recipe object will appear and the game will end. At the bottom left corner of the screen, the instructions of the recipe will be displayed.


\subsection*{Tentative Recipe Details}
\label{sec:orgcd3a2e4}
\begin{itemize}
\item These tentative recipes are based upon objects that we have already implemented in our prototype as well as some additonal features that we will implement in the future

\item Curry
\begin{itemize}
\item Objects: Potato, Tomato, Onion, Knife, Pot
\item Use knife on Potato (whole)
\item Use knife on Tomato (whole)
\item Use knife on Onion (whole)
\item Use Potato (cut) on Pot
\item Use Tomato (cut) on Pot
\item Use Onion (cut) on Pot
\item Select Pot to turn on the stove
\item Curry Complete
\end{itemize}

\item Fruit Salad
\begin{itemize}
\item Objects: Mango, Orange, Banana, Knife
\item Use Knife on Mango (whole)
\item Use Knife on Orange (whole)
\item Use Knife on Banana (whole)
\item Use Mango (cut) on Bowl
\item Use Orange (cut) on Bowl
\item Use Banana (cut) on Bowl
\item Use Spoon on Bowl to mix
\item Fruit Salad Complete
\end{itemize}

\item Steak
\begin{itemize}
\item Objects: Steak, Pan
\item Use Salt on Steak
\item Use Steak on Pan
\item Select Pan to turn on the stove
\item Steak Complete
\end{itemize}
\end{itemize}
\end{document}
